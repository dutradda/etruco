\documentclass[]{article}

\usepackage[utf8x]{inputenc}
\usepackage[brazil]{babel}

\title{Ferramentas de mantenedor}
\author{Diogo Dutra}
\date{22 de Outubro de 2008}

\begin{document}
\maketitle
\tableofcontents

\section{Introdução}
Iremos utilizar ferramentas para o auxílio a manutenção do código do ETruco, as ferramentas a serem utilizadas serão as GNU autotools, tais quais autoconf, automake e libtool, o pkg-config e por último o autogen.sh.

\section{autoconf}
O GNU autoconf é uma ferramenta de auxílio para criar um ambiente portável de configurações e compilação envolta de um software.
Ele quem é o encarregado de criar todos os arquivos para configuração de compilação do software, verificar o que precisa para compilação, dependências, parâmetros, etc.

Basicamente o autoconf lê o arquivo configure.ac e aplica as regras contidas dentro dele.
O arquivo configure.ac é um shell script, pode-se usar as macros pré-estabelecidas pelo autoconf, serão citadas aqui apenas algumas de muita utilidade.
\begin{verbatim}
# Inicia o autoconf
AC_INIT(NOME, VERSAO, BUGS REPORT)

# Iniciar o automake
AM_INIT_AUTOMAKE

# Verifica se PROGRAMA existe,
# normalmente usa-se LIBTOOL, CC ou CXX
AC_PROG_PROGRAMA

# Exporta variáveis para dentro das configurações,
# pode ser usado também pelo automake
AC_SUBST(VARIAVEIS)

# Chama o pkg-config para checar essas dependências
PKG_CHECK_MODULES(NOME MODULO,[MODULOS A SEREM CHECADOS])

# Os headers criados pelo autoconf,
# normalmente só usa-se o config.h
AC_CONFIG_HEADERS(config.h)

# Os arquivos que serão gerados pelo configure
# Os arquivos em ARQUIVOS podem ser arquivos .in
# para serem pré-processados pelo autoconf
AC_CONFIG_FILES(ARQUIVOS)

# Chama a criação dos arquivos
AC_CONFIG_OUTPUT
\end{verbatim}

\section{automake}
O automake é uma ferramenta para se criar Makefiles automáticos seguindo um padrão específico.
Para utilizar o automake corretamente é necessário se ter um arquivo Makefile.am em cada diretório q se deseja utilizar na compilação.
Abaixo são citadas algumas das importantes variáveis q o automake entende.
\begin{verbatim}
# Indica quais são os subdiretórios
# que serão usados, isto é obrigatório caso
# haja um subdiretório que será usado na compilação
SUBDIRS


# ONDE é onde os programas
# executáveis EXECS serão instalados,
# ONDE podem ser diretórios padrões
# bin, lib, etc, ou o pode ser qualquer,
# onde o valor de qualquer será o valor
# da variável qualquerdir
ONDE_PROGRAMS EXECS

# O mesmo de PROGRAMS vale para esse,
# mas aqui serão colocadas as bibliotecas
ONDE_LIBRARIES LIBS

# Segue a mesma métrica de ONDE_PROGRAMS para os demais
# Bibliotecas da libtool
ONDE_LTLIBRARIES

# Cabeçalhos a serem instalados (.h)
ONDE_HEADERS

# Scripts extras
ONDE_SCRIPTS

# Os dados q a aplicação usará
# Caso queira se criar algum arquivo destes citados 
# anteriormente para uso dentro do projeto,
# mas não na instalação usa-se noinst em ONDE
ONDE_DATA


# Os fontes necessários para compilar COISA,
# COISA é um parâmetro passado para 
# ONDE_PROGRAMS e todos anteriormente citados
# Caso um desses parâmetros contenha caracteres
# não-alpha-numéricos deve ser colocado um _ no lugar
COISA_SOURCES

# Flags adicionais para C
COISA_CFLAGS

# Flags adicionais para C++
COISA_CPPFLAGS

# Linkagem de objetos para o caso de COISA ser um executavel
COISA_LDADD

# Linkagem de objetos para o caso de COISA ser uma biblioteca
COISA_LIBADD

# Flags adicionais para o linker
COISA_LDFLAGS

# Arquivos adicionais q deseja ser empacotados
EXTRA_DIST
\end{verbatim}

\section{libtool}
Abstrai todo processo de criação de bibliotecas, sempre criará um arquivo .la, que será compilado posteriormente de acordo com a plataforma q se está utilizando.
Seu uso é bem simples.
Basta habilitar a opção AC\_PROG\_LIBTOOL no autoconf e utilizar sempre o LTLIBRARIES no automake.
Em regras como LDADD e LIBADD sempre apontar para os arquivos .la. Pronto, a libtool já vai fazer todo o processo de criação de bibliotecas.

\section{pkg-config}
Ferramenta utilizada para fazer checagem de dependências, ele utiliza arquivos SEU\_PROGRAMA.pc, o conteúdo deste arquivo tem informações sobre o programa.
Deve conter:
\begin{verbatim}
Name: Nome do programa
Description: Breve descrição do programa
Version: Versão do programa
Require: nome das dependências separados por vírgula
Libs: -Llocal_das_libs -lqual_lib
Libs.private: Dependências básicas dessa lib
Cflags: -Ilocal_dos_cabeçalhos OUTRAS_FLAGS
\end{verbatim}

\section{autogen.sh}
O autoconf possui uma ferramenta chamada autoreconf, que é encarregada de chamar o autoconf, automake e a libtool.
Mas o autoreconf tem algumas limitações, por isso foi criado o autogen.sh, e ele será utilizado no lugar do autoreconf.

Sua utilização é bem simples, o script autogen.sh se encontra no próprio diretório do projeto, e basta executá-lo para fazer as configurações.

\section{Referências}
\begin{verbatim}
http://www.lrde.epita.fr/~adl/autotools.html
\end{verbatim}

\end{document}
